\documentclass[11pt]{article}
\usepackage[margin=1.0in]{geometry}
\usepackage{setspace}
\usepackage{amsmath}
\usepackage{pdflscape}
\usepackage{fancyhdr}
\usepackage{listings}
\usepackage{tikz-qtree,tikz-qtree-compat}

\usepackage{tikz}
\pagestyle{fancy}
\lhead{CS 4610 Written Assignment 3}
\rhead{Zihao Wang (zw2rf)}
\cfoot{\thepage}
\renewcommand{\headrulewidth}{0.4pt}
\renewcommand{\footrulewidth}{0.4pt}
\renewcommand{\thesubsection}{\alph{subsection}.}
\begin{document}
\thispagestyle{empty}
\title{CS 4610 Written Assignment 3}
\author{Zihao Wang (zw2rf)}
\date{\today}
\maketitle
\doublespacing

\section{Convert the following grammars into $LL(1)$ grammars.}
\subsection{}
$$E \rightarrow TP$$
$$P \rightarrow \epsilon \mid +TP \mid !P$$
$$T \rightarrow int \mid (E)$$

\subsection{}
$$L \rightarrow XP$$
$$P \rightarrow \epsilon \mid ,XP$$
$$X \rightarrow int \mid string \mid (L)$$

\subsection{}
$$P \rightarrow p G$$
$$G \rightarrow \epsilon \mid H 4 U G$$
$$H \rightarrow h$$
$$U \rightarrow uQ$$
$$Q \rightarrow \epsilon \mid P$$

\begin{landscape}
\section{}
$$y + + + y + + +$$
\begin{center}
\begin{tabular}{ | c | c | c | c | c | }
\hline
chart[0] & chart[1] & chart[2] & chart[3] & chart[4]   \\
\hline
$S \rightarrow \bullet A$, 0 & $B \rightarrow y \bullet $, 0 &  $A \rightarrow B + \bullet +$, 0 & $A \rightarrow B + + \bullet $, 0 & $A \rightarrow B +++ \bullet A$, 0 
\\
\hline
$A \rightarrow \bullet B++$, 0 & $A \rightarrow B \bullet ++$, 0 & $A \rightarrow B + \bullet ++A$, 0 & $A \rightarrow B ++ \bullet +A$, 0 & $A \rightarrow \bullet B++$, 4
\\
\hline
$A \rightarrow \bullet A + A$, 0 & $A \rightarrow B \bullet +++A$, 0 && $S \rightarrow \bullet A$, 0 & $B \rightarrow \bullet y$, 4
\\
\hline
$B \rightarrow \bullet y$, 0 & & & $A \rightarrow \bullet B++$, 3 &
\\
\hline
$A \rightarrow \bullet B+++A$, 0 &&& $B \rightarrow \bullet y$, 3 & 
\\
\hline

\end{tabular}

\vspace{10 mm}

\begin{tabular}{ | c | c | c | c | c | }
\hline
chart[5] & chart[6] & chart[7] & chart[8]  \\
\hline
$B \rightarrow y \bullet $, 4&$A \rightarrow B +\bullet +$, 4 &$A \rightarrow B ++\bullet $, 4  &  $A \rightarrow A + \bullet A$, 0
\\
\hline
$A \rightarrow B \bullet ++$, 4& & $A \rightarrow B +++ A \bullet $, 0  & 
\\
\hline
& &  $S \rightarrow A \bullet $, 0 &  
\\
\hline
& &  $A \rightarrow A \bullet + A$, 0&  
\\
\hline

\end{tabular}

\vspace{10 mm}

There are more states to be added into the table as the grammar $A \rightarrow A + A$ is left recursive. However, if more recursions are added to the table, we will only have more tokens which would not lead to the acceptance of the string. \\ 
Therefore, these grammars do not accept the string $y+++y+++$

\end{center}
\end{landscape}

\section{Explain in English prose how you would modify an Earley recognizer into a full-fledged parser that executes user actions.}
In order to make a Earley recognizer work as a fully fledged parser, we need information such as line numbers and lexeme for every terminal. For non-terminals, we need to keep track of the identify of that non-terminal. Usually, we have non-terminals representing some important features of a language. For example, in languages such as COOL, we have non-terminals representing each class and method. With these information, we can better group the grammar rules for the users. \\
As for data structures, a dictionary will suffice for a grammar rule. Each field such as line number, arguments for a methods are separate entries in this dictionary. A huge table for the Earley parsing algorithm takes too much memory. We can keep a queue instead. For every token read, we do a shift by keeping those rules for which we can do a shift operation. For every closure, we need to record the original rule as a field in the new rule to prepare for reduction. Line numbers are also sent in a closure. In a reduction operation, all the information about the original rule is sent to the reduced rule. User actions are executed every time a grammar rule is reduced. 

\section{Consider an Earley parser with the closure operation removed:}

\subsection{Give a grammar $G_{1}$ and a string $S_{1}$ such that $S_{1} \in L(G_{1})$ but this parser fails to recognize it.}
$G_{1}$:
$$S \rightarrow Ab$$
$$A \rightarrow a$$
$S_{1}$:
$$ab$$
This grammar fails to reognize the string as the starting rule has other non-terminals in it.

\subsection{Give a grammar $G_{2}$ and a string $S_{2}$ such that $S_{2} \in L(G_{2})$ but this parser still successfully recognizes it.}
$G_{2}$:
$$S \rightarrow ab$$
$S_{2}$:
$$ab$$
This grammar recognizes the string as the starting has all terminals in it. 
\end{document}
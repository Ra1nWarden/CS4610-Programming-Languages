\documentclass[11pt]{article}
\usepackage[margin=1.0in]{geometry}
\usepackage{setspace}
\usepackage{amsmath}
\usepackage{fancyhdr}
\usepackage{mathpartir}
\usepackage{listings}
\usepackage{tikz-qtree,tikz-qtree-compat}
\usepackage{tikz}
\pagestyle{fancy}
\lhead{CS 4610 Written Assignment 7}
\rhead{Zihao Wang (zw2rf)}
\cfoot{\thepage}
\renewcommand{\headrulewidth}{0.4pt}
\renewcommand{\footrulewidth}{0.4pt}
\renewcommand{\thesubsection}{\alph{subsection}.}
\lstset {
	language=C++,
	frame=single,
	numbers=none,
	backgroundcolor=\color{white}
}

\begin{document}
\thispagestyle{empty}
\title{CS 4610 Written Assignment 7}
\author{Zihao Wang (zw2rf)}
\date{\today}
\maketitle
\doublespacing

\section{Consider Stop \& Copy vs. Mark \& Sweep garbage collection}

\subsection{Is one of these two GC algorithms 'faster' than the other? Which algorithm needs to be run more frequently?}
Stop \& Copy algorithm is faster than Mark \& Sweep algorithm. Stop \& Copy needs to be run more frequently as the memory is divided into two equal segments. Stop \& Copy is called when half of the memory is filled up while Mark \& Sweep is run when the whole memory is filled up.

\subsection{Does either algorithm use strictly more memory than the other?}
Yes, Mark \& Sweep uses strictly more memory than Stop \& Copy due to the mark bits.

\subsection{Are reference cycles common in everyday data structures?}
Normally, reference cycles are not common in everyday data structures.

\subsection{Briefly describe how one might implement a cycle detector. When can a cycle be cleaned?}
To implement a cycle detector, we can use a tracing algorithm starting from directly reachable references and mark all the references that are reachable. The unmarked ones with reference count more than 0 are included in a reference cycle. This is similar to the mark phase of Mark \& Sweep. It should be called periodically. 

\section{Suppose we want to add a new construct to the language: \lstinline|protect e|}

\subsection{Give the new type rules for \lstinline|protect|, \lstinline|try/catch| and \lstinline|throw|}
\lstinline|throw|
$$
\inferrule*
	{O, M, C \vdash e : T_{1}, E_{1}}
	{O, M, C \vdash throw \text{ } e : T_{2}, true}
$$
\lstinline|try/catch|
$$
\inferrule*
	{O, M, C \vdash e_{0} : T_{0}, E \qquad O\left[ Object/ x \right],M, C \vdash e_{1} : T_{1}, E'}
	{O, M, C \vdash try \text{ } e_{0} \text{ } catch \text{ } x : Object \Rightarrow e_{1} : T_{0} \sqcup T_{1}, false}
$$
\lstinline|protect|
$$
\inferrule*
	{O, M, C \vdash e : T, false}
	{O, M, C \vdash protect \text{ } e : T, false}
$$

\subsection{Give a code sample that illustrates that the typing system described here is not necessarily safe.}
\begin{lstlisting}
protect {
	try {
		throw e
	} catch x {
		throw e
	}
}
\end{lstlisting}
This codes passes the type check as the \lstinline|try/catch| block returns false since it catches all types of exceptions. Therefore, throwing an exception in the \lstinline|catch| blocks fools the compiler to type check the program. 

\end{document}